\documentclass[11pt,utf8,notoc,bibnum,german,final]{zihpub}
\usepackage{url}

\title{Escalator}
\author{Hermann Loose}
\matno{3411242}
\betreuer{Jupp Müller}
\bibfiles{report}
\begin{document}

% TODO(hermannloose): Was ist ein gutes Wort dafür?
\section{Related Work}

Gesucht wurde nach einer \emph{flexiblen, quell-offenen}
Benachrichtigungslösung, welche zudem \emph{rotierende Bereitschaftsdienste}
verwaltet und möglichst auch \emph{nur} das tut. Es folgen einige kurz
evaluierte Optionen.

\paragraph{Nagios Benachrichtigungen \cite{nagios-notifications}}

Nagios ermöglicht die Definition von Eskalationsstrategien für
Benachrichtigungen über Konfigurationsdateien. Weitere Recherche förderte für
diese Methode kein zufriedenstellendes Web-Frontend zutage. Das einzige
Resultat in dieser Richtung war NoMa \cite{noma}, welches nicht die gewünschte
Flexibilität im Bezug auf die Verwaltung von Eskalationsstrategien und die
Rotation von Bereitschaftsdiensten zu bieten schien.

Zudem würde sich eine Lösung mittels Nagios gleichzeitig auf Nagios beschränken
und dem Wunsch gegenüber stehen, die Benachrichtigungen verschiedener Dienste
in einem einzelnen System aggregieren zu können.

\paragraph{PagerDuty \cite{pagerduty}}

In San Francisco ansässiges Start-up, dessen Produkt der Verkauf von Anruf- und
SMS-Guthaben rund um ein Benachrichtigungssystem ist, welches den Vorstellungen
für Escalator—ob der relativen Offensichtlichkeit dieser Idee—sehr nahe kommt.
Allerdings fällt hier die fehlende Quelloffenheit ins Gewicht, welche es
unwahrscheinlich macht, das System um die angedachte Android-Integration
erweitern zu können.

Zudem scheint das Routing von Ereignissen aus Monitoringsystemen zu PagerDuty
ausschließlich über Email abzulaufen, was für eine verbesserte Anbindung der
Daten zuführenden Systeme in der Zukunft über eine robuste Schnittstelle—im
Falle von Escalator HTTP nach REST-Prinzipien—ungünstig erscheint.

\paragraph{}

Da sich keine der betrachteten Optionen als im Bezug auf die genannten
Kriterien zufriedenstellend erwies, wurde beschlossen, eine prototypische
Anwendung für eben diesen Zweck selbst zu implementieren.

\section{Implementierung}

\subsection{Web-Schnittstelle und Backend}

Für die Implementierung der Web-Schnittstelle und des Backends wurde das
Web-Framework Rails \cite{rails} ausgewählt, da ich bereits mit der Sprache
Ruby vertraut war und die Ruby on Rails Guides \cite{rails-guides} sehr viel
unterstützendes Material für Anfänger bereit stellen.

\subsection{Android-App}
