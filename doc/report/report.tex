\documentclass[11pt,utf8,notoc,bibnum,german,final]{zihpub}
\usepackage{url}

\title{Escalator}
\author{Hermann Loose}
\matno{3411242}
\betreuer{Jupp Müller}
\bibfiles{report}

\begin{document}

\section{Ausgangspunkt}

Vor dem Beginn dieser Arbeit stand ursprünglich die Idee, ein mobiles Frontend
für ein Monitoringsystem—im konkreten Fall \emph{Ganglia}~\cite{ganglia},
dieser Vorschlag stammte von Sven Jansky (Praktikumspartner)—zu
entwickeln, verbunden mit der Möglichkeit, bei bestimmten Ereignissen
Push-Benach\-richti\-gungen auf mobilen Endgeräten zum empfangen.

Hinzu kam der Wunsch, z.B. den Status der Warteschlangen des Batchsystems mit
einzubeziehen, etc.; alles möglichst mit der Option, direkt vom mobilen
Endgerät mittels einiger einfacher, vorgeschlagener Aktionen auf die jeweiligen
Umstände reagieren zu können.

Hier zeichneten sich Benachrichtigung und Reaktion als zwei Problemfelder ab,
die besser separat voneinander zu lösen sind. Mit der Reaktion als stärker
anwendungsspezifischem Teil und der Benachrichtigung als dem eher generischen,
fiel die Wahl der Aufgabenstellung auf letzteren, da diese Richtung im Umfang
überschaubarer schien und hier schneller ein praktischer Nutzen zu erwarten
ist.

% TODO(hermannloose): Was ist ein gutes Wort dafür?
\section{Existierende Lösungen}

Gesucht wurde nach einer \emph{flexiblen, quell-offenen}
Benachrichtigungslösung, welche zudem \emph{rotierende Bereitschaftsdienste}
verwaltet und möglichst auch \emph{nur} das tut. Weiter lag der Fokus auf
\emph{einfacher Erweiterbarkeit} und der Verfügbarkeit einer
\emph{REST-Schnittstelle} zum Anlegen bzw. Empfangen von Ereignissen. Es folgen
einige kurz evaluierte Optionen.

\paragraph{Nagios Benachrichtigungen~\cite{nagios-notifications}}

\emph{Nagios} ermöglicht die Definition von Eskalationsstrategien für
Benachrichtigungen über Konfigurationsdateien. Weitere Recherche förderte für
diese Methode kein zufriedenstellendes Web-Frontend zutage. Das einzige
Resultat in dieser Richtung war \emph{NoMa}~\cite{noma}, welches nicht die gewünschte
Flexibilität im Bezug auf die Verwaltung von Eskalationsstrategien und die
Rotation von Bereitschaftsdiensten zu bieten schien.

Zudem würde sich eine Lösung mittels Nagios gleichzeitig auf Nagios beschränken
und dem Wunsch gegenüber stehen, die Benachrichtigungen verschiedener Dienste
in einem einzelnen System aggregieren zu können.

\paragraph{PagerDuty~\cite{pagerduty}}

In San Francisco ansässiges Start-up, dessen Produkt der Verkauf von Anruf- und
SMS-Guthaben rund um ein Benachrichtigungssystem ist, welches den Vorstellungen
für Escalator—ob der relativen Offensichtlichkeit dieser Idee—sehr nahe kommt.
Allerdings fällt hier die fehlende Quelloffenheit ins Gewicht, welche es
unwahrscheinlich macht, das System um die angedachte Android-Integration
erweitern zu können.

Zudem scheint das Routing von Ereignissen aus Monitoringsystemen zu PagerDuty
ausschließlich über Email abzulaufen, was für eine verbesserte Anbindung der
Daten zuführenden Systeme in der Zukunft über eine robuste Schnittstelle—im
Falle von Escalator HTTP nach REST-Prinzipien—ungünstig erscheint.

\paragraph{Fazit}

Da sich keine der betrachteten Optionen als im Bezug auf die genannten
Kriterien zufriedenstellend erwies, wurde beschlossen, eine prototypische
Anwendung für eben diesen Zweck selbst zu implementieren.

\section{Konzept}
\label{concept}

Eingehende Ereignisse werden in Escalator zunächst jeweils einem von
beliebig vielen, frei definierbaren Zuständigkeitsbereichen zugeordnet. Diese
Unterteilung kann beispielsweise anhand der betroffenen Technologie erfolgen,
wie etwa \emph{Webserver}, \emph{Datenbank}, \emph{Netzwerk}, oder auch
geographisch motiviert sein, wie etwa \emph{Data Center A}, \emph{Data Center
B}, etc.

Innerhalb eines solchen Zuständigkeitsbereichs soll eine einfache Hierarchie
von zu benachrichtigenden Benutzern abgebildet werden, welche im Bezug auf
diesen Bereich \emph{Bereitschaftsdienst} haben. Die erste Ebene des
Bereitschaftsdienstes ist intuitiv über eingehende Ereignisse immer ohne
zeitliche Verzögerung zu benachrichtigen. Erfolgt von Seiten des betroffenen
Benutzers keine Reaktion auf ein Ereignis, wird nach einer für den jeweiligen
Zuständigkeitsbereich definierten Zeitspanne die nächste Ebene des
Bereitschaftsdienstes benachrichtigt.

Auf diesem Weg soll von Escalator als Kernfunktionalität sichergestellt werden,
dass auftretende Ereignisse nie über längere Zeit unbeachtet bleiben, aus
vielfältigen Gründen wie „Diensthandy im Funkloch bzw. aufgrund niedrigen
Akkuladezustands ausgeschaltet“, „Bereitschaftsdienst überhört Diensthandy“,
etc. Darüber hinaus ergeben sich wünschenswerte Features, die teilweise im
Prototyp bereits erprobt wurden.

Erhält ein Bereitschaftsdienst eine Benachrichtigung, die er zwar registriert,
aufgrund der Umstände allerdings aktuell nicht bearbeiten kann, bietet
Escalator im Prototyp—für Android als Plattform des mobilen Endgeräts—bereits
die Möglichkeit, dieses Ereignis sofort manuell zur nächsten Hierarchieebene zu
eskalieren, um die geringstmögliche Verzögerung bis zur tatsächlichen
Behandlung eines Ereignisses zu erreichen.

Ebenfalls prototypisch angelegt ist die automatische, regelmäßige Rotation der
Benutzer, welche sich als Gruppe den Posten eines Bereitschaftsdienstes teilen.
Die vorgesehene Nutzung des Features ist hier z.B. ein wöchentlicher Wechsel
des zuständigen Benutzers. Der konkrete Zeitplan ist für jede einzelne
Hierarchieebene separat konfigurierbar.

Da wahrscheinlich jeder Benutzer eigene Präferenzen für die Art und Weise der
Benachrichtigung haben wird, besitzt der Prototyp bereits die Möglichkeit, für
jede Mitgliedschaft in einer solchen Gruppe einzeln festzulegen, in welchen
Zeitabständen er über welche Wege benachrichtigt werden soll. Erfolgt so z.B.
auf eine unmittelbar nach Zuteilung eines Ereignisses versandte SMS nicht
innerhalb von drei Minuten eine Reaktion, versendet Escalator eine Email, bzw.
es wird ein automatischer Anruf getätigt etc.

Auf diesem Weg kann der zuständige Benutzer auch bei Ausfall oder
Nichtbeachtung eines Benachrichtungsweges zuverlässig—sofern er diese Schritte
sinnvoll gewählt hat—erreicht werden, ohne dass im Normalfall zuviele
gleichzeitige Benachrichtigungen über zuviele verschiedene Kanäle für eine
Abstumpfung gegenüber diesen Signalen sorgen. Ebenso ist eine intuitive
Abgrenzung von Zuständigkeitsbereichen anhand des Benachrichtigungsweges für
ein Ereignis möglich, falls der Benutzer mehrere Bereitschaftsdienste
gleichzeitig erfüllen sollte und dies wünscht.

% FIXME(hermannloose): Eventuell weiter unten als "Ausblick"?

Geplant ist weiterhin die Möglichkeit der flexiblen Definition von
Urlaubszeiträumen, Krankschreibungen etc. für jeden Benutzer, sodass von
Escalator in dieser Zeit andere Benutzer ersatzweise zum Bereitschaftsdienst
eingeteilt werden können. Idealerweise sollte diese Möglichkeit zu einem hohen
Grad in der Hand der betroffenen Benutzer liegen, um eine zusätzliche
Arbeitsbelastung von Benutzern mit administrativer Rolle gering zu halten. Dies
würde auch spontanen Diensttausch unter Benutzern unbürokratisch gestalten.

\section{Implementierung}

\subsection{Web-Schnittstelle und Backend}

Für die Implementierung der Web-Schnittstelle und des Backends wurde das
Webframework \emph{Rails}~\cite{rails} ausgewählt, da ich bereits mit der Sprache
Ruby vertraut war, um Rails als praxiserprobtes Framework herum ein sehr
fortgeschrittenes Ökosystem von Erweiterungen existiert und die \emph{Ruby on Rails
Guides}~\cite{rails-guides} sehr viel unterstützendes Material für Anfänger
bereit stellen.

Aufgrund seiner Ausrichtung auf den REST-Architekturstil~\cite{fielding-rest}
eignet sich Rails dazu, sowohl das Web-Frontend für die Verwaltung von
Eskalationsstrategien, Rotationen, Benutzern etc. als auch die API für die
Anbindung von Drittsystemen, welche Ereignisse liefern, in einer einzigen
Anwendung zu vereinen. Codegenerierung mittels \emph{Scaffolding} ermöglichte
hier einen sehr schnellen Start bei der Anlage grundlegender Strukturen der
Datenhaltung und von Formularen zur Manipulation von diesen.

\paragraph{Ermittlung der Zuständigkeit}

Die in Abschnitt \ref{concept} eingeführten Zuständigkeitsbereiche werden
innerhalb der Anwendung als \emph{Escalation Policies} abgebildet. Die in den
Bereich fallenden Ereignisse sind in der Form von \emph{Issues} mit diesem
assoziiert. Eine Escalation Policy enthält weiterhin beliebig viele
\emph{Escalation Steps}, die festhalten, nach welcher zeitlichen Verzögerung
welcher Benutzer benachrichtig werden sollen.

Zu diesem Zweck verweisen die Escalation Steps auf sogenannte \emph{Rotations},
in denen eine Gruppe von Benutzern über \emph{Rotation Memberships} Mitglied
ist. Jede dieser Memberships trägt zugleich einen Rang, über den festgelegt
ist, welcher Benutzer aktuell Bereitschaftsdienst hat. Der Wechsel des
diensthabenden Benutzers erfolgt im Prototyp mittels eines zu festgelegten
Zeitpunkten laufenden \emph{Rotation Job}s, welcher die Ränge der
Mitgliedschaften einer Rotation umlaufend aktualisiert.

\paragraph{Benachrichtigung}

Ist der zuständige Benutzer für ein neu eingegangenes Ereignisse über diesen
Weg festgestellt, wird über die in einer Rotation Membership enthaltenen
\emph{Alerting Steps} bestimmt, wie der Benutzer zu benachrichtigen ist. Dazu
besitzt jeder Alerting Step einen Parameter, der die zeitliche Verzögerung
angibt, nach der dieser Alerting Step aktiv wird.

Das dort ebenfalls definierte \emph{Contact Detail} enthält in
Schlüssel/Wert-Paaren verschiedene Parameter, welche der mit dem Typ des
Contact Details assoziierte \emph{Service} benötigt, um die Informationen über
das Ereignis für den jeweiligen Endpunkt aufzubereiten und den korrekten
Versand der Nachricht zu garantieren.

Aktuell implementiert sind Services für Emailversand und
Push-Benachrichtigungen an Android-Geräte. Für neue Benachrichtigungsendpunkte
kann dieser Teil einfach und schnell erweitert werden

% TODO(hermannloose): Left off here.

\paragraph{Eingesetzte Techniken}

Während der Entwicklung des Prototyps habe ich mich neben der eigentlichen
Programmierung mit einigen \emph{Best Practices} der Rails-Entwicklung befasst.
So erfolgte nach der anfänglichen Verwendung des in Ruby selbst enthaltenen
Testframeworks \emph{Test::Unit}~\cite{testunit}—bei der Neuanlage von
Rails-Anwendungen ohne weitere Parameter standardmäßig ausgewählt—nach kurzer
Zeit der Wechsel zum mittlerweile Quasi-Standard der Rails-Welt in dieser
Hinsicht, \emph{RSpec}~\cite{rspec}.

Generell lag der Fokus darauf, zum Zeitpunkt der Abgabe eine prototypische
Anwendung präsentieren zu können, die neben der geplanten Funktionalität auch
eine möglichst hohe Testabdeckung vorzuweisen hat, um eine Weiterentwicklung
attraktiv zu machen.

In diesem Rahmen wurde auch die Verwendung von \emph{Cucumber}~\cite{cucumber}
erprobt, einem Framework für \emph{Behaviour-Driven Development}, mit dem für
den Nutzer sichtbare Features des Systems in Textform beschrieben und darauf
aufbauend automatisiert getestet werden können.

Für die asynchrone Abarbeitung von im Hintergrund laufenden
Tasks—beispielsweise die rechtzeitige Eskalation von Ereignisse nach Ablauf
einer Zeitspanne—wurde die für diese Zwecke populäre Erweiterung
\emph{delayed\_job}~\cite{delayedjob} eingesetzt. Diese hat den Vorteil, dass
die Jobverwaltung komplett über die bereits von Rails verwendete Datenbank
abgewickelt werden kann und die Einstiegshürde sehr gering ist.

Da es sich bei Escalator weniger um einen Wegwerf-Prototypen als vielmehr um
Entwicklung nach dem Prinzip von \emph{Tracer
Bullets}~\cite[S. 47 ff.]{pragprog-tracerbullets} handeln soll, wurden Authentifizierung
und Authorisierung innerhalb der Anwendung ebenfalls beleuchtet, um auch unter
diesen Gesichtspunkten zu planen.  Die Authentifizierung erfolgt mittels
\emph{Devise}~\cite{devise}, die Authorisierung auf Ressourcenebene
rollenbasiert via
\emph{declarative\_authorization}~\cite{declarative-authorization}, beides
ebenfalls sehr populäre Lösungen in Form von Rails-Erweiterungen.

% TODO(hermannloose): Deployment erwähnen.

\subsection{Android-App}

Für mobile Geräte ab Android 2.2—Entwicklung und Test erfolgten auf einem
privaten \emph{HTC Desire}—wurde begleitend eine einfache Anwendung zum Empfang
von Push-Benachrichtigungen entwickelt, unter Verwendung des Android
\emph{Cloud to Device Messaging} (C2DM) Frameworks~\cite{c2dm}.

\paragraph{Anmeldung}

Die Anwendung muss dazu den unter \cite{c2dm-lifecycle} beschriebenen Ablauf
zur Anmeldung für diesen Dienst implementieren. Nach der vom Gerät initiierten
Anmeldung an Googles C2DM Server erhält die Anwendung eine sogenannte
\emph{Registration ID}, welche für den Nachrichtenversand von der bei der
Anmeldung angegeben Quelle zu diesem Gerät gültig ist. Dieser Anmeldevorgang
kann im Prototyp zu Testzwecken explizit ausgelöst werden, in der endgültigen
Anwendung sollte der Prozess für den Benutzer transparent im Hintergrund
ablaufen. Weiterhin muss die Anwendung auf die Vergabe einer neuen Registration
ID durch Google selbstständig reagieren, was laut \cite{c2dm-lifecycle} von
Zeit zu Zeit (ohne konkrete Angabe wie oft) passieren kann.

Besitzt die Anwendung eine solche gültige Registration ID, muss diese an das
Escalator-Backend weitergereicht werden, damit von dort der Nachrichtenversand
erfolgen kann. Zum einen ist hierfür die Authentifizierung am Backend
notwendig, welche über ein in der Android-App gespeichertes
Authentifizierungstoken durchgeführt wird. Im zweiten Schritt muss die
Registration ID einem existierenden Contact Detail vom Typ Android zugeordnet,
bzw. ein neues solches Contact Detail erstellt werden.

Da aus Zeitgründen während der Entwicklung des Prototyps auf Android auf eine
Oberfläche zur Auswahl eines existierenden Contact Details verzichtet wurde—die
dafür notwendige API ist jedoch bereits vorhanden—legt die Android-App bei
diesem Vorgang zu Demonstrationszwecken immer ein neues Contact Detail mit der
aktuellen Registration ID an. In der Weboberfläche kann dieses Contact Detail
dann einem Alerting Step zugeordnet werden, um bei Bereitschaftsdienst in der
jeweiligen Rotation über das Gerät benachrichtigt zu werden.

\paragraph{Benachrichtigung}

Soll nun ein Android-Gerät über ein Ereignis benachrichtigt werden, sendet das
Escalator-Backend den unter \cite{c2dm-server} beschriebenen HTTP Request an
den C2DM Server.

Das Gerät empfängt daraufhin im Normalfall vom C2DM Server einen
\texttt{RECEIVE}-Intent mit den Parametern des ursprünglichen Ereignisses als
Payload. Der Benutzer wird mittels eines einfachen Textes in der Statusleiste
von Android auf das neue Ereignis hingewiesen. Ein Klick auf diese
Benachrichtigung öffnet daraufhin eine einfache Ansicht mit den Details des
Ereignisses.

Innerhalb dieser Detailansicht hat der Benutzer im Prototyp bereits die
Möglichkeit, durch Drücken des „Acknowledge“-Buttons das Ereignis zu
akzeptieren, was eine weitere Eskalation verhindert. Verlässt der Benutzer die
Detailansicht, existiert aktuell kein Weg, das Ereignis und damit den
„Acknowledge“-Button erneut aufzurufen. Eine entsprechende Erweiterung der
Anwendung sollte aber unkompliziert ausfallen.

% TODO(hermannloose): Left off here.

\section{Ausblick}

% TODO(hermannloose): Angedachte Features auflisten.

\paragraph{Weiterentwicklung}

Der Code für Escalator ist öffentlich auf \emph{GitHub}~\cite{escalator-github}
verfügbar. Die Wahl einer passenden Open-Source-Lizenz steht noch aus, wird
aber vermutlich auf eine Form der MIT-Lizenz fallen. Eine Weiterentwicklung ist
für jeden Interessenten möglich, ein GitHub-Account wird hierfür nicht benötigt.

\end{document}
